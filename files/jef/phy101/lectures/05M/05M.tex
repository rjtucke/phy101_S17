\documentclass[english]{beamer}
\usepackage[per-mode=fraction]{siunitx}
\useoutertheme{RJTcustom}

\begin{document}

\begin{frame}{Lecture 05M Plan}
  \textbf{Material}
  \begin{itemize}
    \item Conservation
    \item Momentum
    \item Impulse
    \item Collisions
    \item NTQ for Wednesday
  \end{itemize}
\end{frame}

\begin{frame}{Conservation}
  \begin{itemize}
    \item Conservation means something doesn't change.
    \item Conservation of momentum means final momentum - initial momentum = 0.
    \item When an external force acts, final momentum - initial momentum = impulse.
  \end{itemize}
\end{frame}

\begin{frame}{Momentum}
  \begin{itemize}
    \item A heavy truck is harder to stop than a small car moving at the same speed.
    \item If two cars have the same mass, the faster one is harder to stop than the slower one.
    \item Define $\textbf{momentum} \equiv \textbf{mass} \times \textbf{velocity}$.
    \item In symbols: $\vec{p} \equiv m \, \vec{v}$.
  \end{itemize}
\end{frame}

\begin{frame}{Impulse}
  \begin{itemize}
    \item Newton's Second Law is about cause and effect:
    \item \begin{itemize}
      \item $\frac{\Sigma \vec{F}}{m}$ is the cause.
      \item $\vec{a}$ is the effect.
    \end{itemize}
    \item Impulse-momentum theorem is the same way:
    \item \begin{itemize}
      \item Impulse $\vec{I}$ is the cause.
      \item A change of momentum $\Delta \vec{p}$ is the effect.
    \end{itemize}
  \end{itemize}
\end{frame}

\begin{frame}{Collisions}
  \begin{itemize}
    \item Again, Newton's 3rd Law: \textbf{you cannot touch without being touched} and both bodies in an interaction feel the same force.
    \item This fact, plus conservation of momentum, is VERY useful in collisions.
    \item If the interaction force is bigger than other forces, momentum is approximately conserved in a collision.
  \end{itemize}
\end{frame}

\begin{frame}{NTQ for Wednesday}
  A cart with a mass of $\SI{0.5}{\kilogram}$ on a frictionless track moves at $\SI{1.0}{\meter\per\second}$ to the right. It strikes a $\SI{0.8}{\kilogram}$ cart, which was moving at $\SI{1.2}{\meter\per\second}$ to the left. What is the direction and magnitude of the system's momentum?
\end{frame}

\end{document}
