\documentclass[english]{beamer}
\useoutertheme{RJTcustom}

\begin{document}

\begin{frame}{Lecture 03W Plan}
  \textbf{Administrative}
  \begin{itemize}
    \item Any licensed radio amateurs in class?
  \end{itemize}
  \textbf{Material}
  \begin{itemize}
    \item NTQ 02W
    \item Newton's Third Law
  \end{itemize}
\end{frame}

\begin{frame}{Next Time Question 02W}
  When I push a refrigerator across a kitchen floor at constant speed,
  the magnitude of the force of friction between refrigerator and floor is\\
  \begin{itemize}
    \item Less than my push force
    \item \textbf{Equal to my push force}
    \item Greater than my push force
  \end{itemize}
  Why? Without acceleration, Newton's First Law guarantees that the \emph{net} force is zero, so the friction must equal the push.
\end{frame}

\begin{frame}{Newton's Third Law}
  Newton's 3rd Law: \textbf{you cannot touch without being touched}.
  \visible<2->{
    \begin{itemize}
      \item Every force is an interaction between two bodies.
      \item Both feel this force the same (though in opposite directions).
      \item Distinguish between ``both feel \emph{this} force the same'' and ``both feel the same \emph{net} force''.
    \end{itemize}
  }
\end{frame}

\begin{frame}{Newton's Third Law}
  Let's identify the force and both bodies in the following examples:
  \begin{itemize}
    \item An instructor walks across a classroom.
    \item A car accelerates along a road.
    \item Two cars collide head-on.
    \item A hammer hits a nail.
    \item An athelete lifts a barbell.
  \end{itemize}
\end{frame}

\begin{frame}{Newton's Third Law}
  Since action and reaction forces are equal and in opposite directions, why don't they cancel?
\end{frame}

\begin{frame}{NTQ 03W}
  A mosquito is struck by the windshield of a car. The force of the impact of the mosquito is \_\_\_\_ the force on the car and the acceleration of the mosquito is \_\_\_\_ the acceleration of the car.
  \begin{itemize}
    \item less than; less than
    \item more than; less than
    \item the same as; less than
    \item the same as; more than
  \end{itemize}
\end{frame}

\end{document}
