\documentclass[english]{beamer}
\usepackage[per-mode=fraction]{siunitx}
\useoutertheme{RJTcustom}

\begin{document}

\begin{frame}{Lecture 14M Plan}
  \textbf{Material}
  \begin{itemize}
    \item Historical Notes
    \item Ernest Rutherford's Discovery
    \item Niels Bohr's Theory
    \item Erwin Schr{\"o}dinger's Equation
    \item NTQ for Wednesday
  \end{itemize}
\end{frame}

\begin{frame}{Hysterical Notes}
  Today has three historical notes:
  \begin{itemize}
    \item Ernest Rutherford discovered the atomic nucleus in 1911.
    \item Niels Bohr gave a quantum theory of the hydrogen atom in 1913.
    \item Erwin Schr{\"o}dinger developed his wave equation in 1926.
  \end{itemize}
\end{frame}

\begin{frame}{Rutherford's Discovery of the Nucleus}
  \begin{itemize}
    \item Rutherford shot particles at a thin sample and observed their deflection.
    \item Most particles were undeflected.
    \item Some particles bounced directly backward.
    \item This shows that nearly all of the mass of the atom is in a nucleus.
    \item For reference: atoms are about $10^{-10}$ m wide and nucleii are about $10^{-15}$ m wide.
  \end{itemize}
\end{frame}

\begin{frame}{Bohr's Model of the Hydrogen Atom}
  \begin{itemize}
    \item Energies in atoms are discrete, not continuous.
    \item Spectral lines reveal energy gaps between discrete states.
    \item Light is emitted when an electron jumps from an excited state to a lower energy state.
    \item These patterns match up with an earlier prediction by Max Planck- the energy carried by a particle of light is $E=h\,f$ where $h$ is a constant.
    \item This model also explain why electrons don't simply spiral into the nucleus. (Why?)
  \end{itemize}
\end{frame}

\begin{frame}{Schr{\"o}dinger's Equation}
  \begin{itemize}
    \item This is not the only `fundamental' quantum mechanics equation.
    \item It marries three concepts:
    \begin{itemize}
      \item quantization,
      \item energy conservation, and
      \item wave mechanics
    \end{itemize}
    \item Basic variable in this equation is $\psi$, the wavefunction.
    \item $\left|\psi\right|^2$ is the probability density for a particle. 
  \end{itemize}
\end{frame}

\begin{frame}{NTQ for Wednesday}
  Explain in a short answer: Why don't electrons spiral into the attracting nucleus?
\end{frame}

\end{document}
