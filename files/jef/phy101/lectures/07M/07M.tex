\documentclass[english]{beamer}
\usepackage[per-mode=fraction]{siunitx}
\useoutertheme{RJTcustom}

\begin{document}

\begin{frame}{Lecture 07M Plan}
  \textbf{Material}
  \begin{itemize}
    \item Historical Note
    \item Overview of battery + bulb experiments
    \item Water analogy for electricity
    \item NTQ for Wednesday
  \end{itemize}
\end{frame}

\begin{frame}{Hysterical Note}
  TBD
\end{frame}

\begin{frame}{Overview of battery + bulb experiments}
  \begin{itemize}
    \item Electricity only flows through components connected in a complete circuit. (Both sides of the battery must be connected and both sides of the bulb must be connected in order to force the electricity to flow through the bulb.)
    \item Current is not used up when it flows around a circuit. Current can split into different branches but the total amount in a circuit doesn't change.
    \item Adding more bulbs in series makes the total current go down (total resistance increases).
    \item Adding more bulbs in parallel makes the total current go up (total resistance decreases).
  \end{itemize}
\end{frame}

\begin{frame}{Water analogy}
  \begin{itemize}
    \item Water flow is an excellent analogy to understand current.
    \item Imagine a mill: a water wheel on a river drives a millstone to grind flour.
    \item The kinetic energy of the river is converted by the water wheel to a different form (grinding flour).
    \item Two important factors affect this energy:
    \item \begin{itemize}
      \item speed of river
      \item depth / width of river
      \end{itemize}
    \item Flow rate is not just velocity, it's also the size of river!
  \end{itemize}
\end{frame}

\begin{frame}{Water analogy}
  \begin{itemize}
    \item 
  \end{itemize}
\end{frame}

\begin{frame}{NTQ for Wednesday}
  Consider a circuit with three identical bulbs: one in series with a 1.5 V battery, then a parallel circuit of the other two bulbs. 
  \begin{enumerate}
    \item How much of the total current flows through each bulb? 
    \item How much resistance does a parallel combination of two bulbs present, relative to a single bulb?
  \end{enumerate}
\end{frame}

\end{document}
