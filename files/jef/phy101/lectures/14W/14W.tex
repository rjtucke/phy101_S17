\documentclass[english]{beamer}
\usepackage[per-mode=fraction]{siunitx}
\useoutertheme{RJTcustom}

\begin{document}

\begin{frame}{Lecture 14W Plan}
  \textbf{Material}
  \begin{itemize}
    \item Historical Note
    \item Ernest Rutherford's Discovery
    \item Niels Bohr's Theory
    \item Erwin Schr{\"o}dinger's Equation
    \item NTQ for Wednesday
  \end{itemize}
\end{frame}

\begin{frame}{Hysterical Note}
  Albert Einstein published four important discoveries (including Special Relativity) in 1905.
\end{frame}

\begin{frame}{Special vs General Relativity}
  \begin{itemize}
    \item Einstein actually invented two theories with `Relativity' in their name.
    \begin{itemize}
      \item Special Relativity (1905) connects space and time into a single `fabric'. It predicts time dilations and length contractions for moving objects. It also contains the famous equation $E=m\,c^2$.
      \item General Relativity (1914) explains that the force of gravity is an illusion. Spacetime is not built out of classical geometry, so straight lines can curve around into circles.
    \end{itemize}
  \end{itemize}
\end{frame}

\begin{frame}{Time Dilation}
  \begin{itemize}
    \item We can deduce the effects of special relativity from the postulate that light moves at the same rate, independent of the observer.
    \item As shown in lab, one consequence of this postulate is that time slows down in a moving reference frame.
    \item The rate of slow-down is given by the \textit{Lorentz factor}, $gamma$.
    \item The Lorentz factor is $\gamma = \left( 1 - \frac{v^2}{c^2} \right)^{-\frac{1}{2}}$, where $v$ is the velocity of the moving frame and $c=3\times10^8$ m/s is the speed of light.
  \end{itemize}
\end{frame}

\begin{frame}{Length Contraction}
  \begin{itemize}
    \item Another consequence of special relativity is that as time expands, distances contract.
    \item It can be shown that length contraction is reciprocal to time dilation. That is, the contraction factor is $\frac{1}{\gamma}$.
  \end{itemize}
\end{frame}

\begin{frame}{NTQ for Monday}
  If you live your entire life on a jet, how much does your total lifespan increase, compared to a stationary observer? Suppose the jet's velocity is 250 m/s and you live $2\times10^9$ s.
\end{frame}

\end{document}
