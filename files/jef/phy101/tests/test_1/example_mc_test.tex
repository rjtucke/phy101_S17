%% LyX 2.1.3 created this file.  For more info, see http://www.lyx.org/.
%% Do not edit unless you really know what you are doing.
\documentclass[english]{exam}
\usepackage[T1]{fontenc}
\usepackage[latin1]{inputenc}
\usepackage{float}
\usepackage{amstext}
\usepackage{graphicx}
\usepackage{setspace}
\usepackage{units}
\onehalfspacing

\makeatletter
%%%%%%%%%%%%%%%%%%%%%%%%%%%%%% Textclass specific LaTeX commands.

            

%%%%%%%%%%%%%%%%%%%%%%%%%%%%%% User specified LaTeX commands.
%% LyX 1.4.2 created this file.  For more info, see http://www.lyx.org/.
%% Do not edit unless you really know what you are doing.



\usepackage{float,calc,color}

\makeatletter

%%%%%%%%%%%%%%%%%%%%%%%%%%%%%% LyX specific LaTeX commands.
%% A simple dot to overcome graphicx limitations

%% Frontmatter vars
\def\thetitle#1{\gdef\@thetitle{#1}}
\def\thedate#1{\gdef\@thedate{#1}}




% ******************************************************************
% *** SPACE CONTROL ************************************************

\def\vf{\vfill}
\def\vs{\vspace{1pc}}
\def\hs{\hspace{1in}}

\def\blank{$\,$ \underline{\hspace{1in}} $\,$}
\def\Blank{$\,$ \underline{\hspace{1.5in}} $\,$}

\def\cpg{\clearpage}

% ******************************************************************
% *** FORMATTING ***************************************************

\def\und{\underline}

% ******************************************************************
% *** ENGLISH ABBREVIATIONS ****************************************

\def\ie{\mbox{{\it i.e.} }}
\def\eg{\mbox{{\it e.g.} }}
\def\etc{\mbox{{\it et cetera} }}

% ******************************************************************
% *** MATH SHORTCUTS ***********************************************

\def\ds{\displaystyle}

\def\ddx{\ds \frac{d}{dx}}
\def\ddt{\ds \frac{d}{dx}}
\newcommand{\dd}[2]{\ds \frac{d{#1}}{d{#2}}}

\def\pitch{\mathbin{\frown \! \! \! \! \mid \, \, \,}}  % replace by \pitchfork if amssymb is working

\def\qed{\rule[-.23ex]{1.6ex}{1.6ex}}

% ******************************************************************
% *** TABLE SHORTHAND **********************************************

\newcommand{\gotable}[3]
      {\begin{table}[h] 
       \begin{center}
       \addtolength{\tabcolsep}{#1mm}
       \renewcommand{\arraystretch}{#2}
       \begin{tabular}{#3}}

\newcommand{\stoptable}
      {\end{tabular}
       \end{center}
       \end{table}}
       
% ******************************************************************
% *** MINIPAGE SHORTHAND *******************************************

\newcommand{\twofig}[4]
   {\begin{figure}[h]
      \begin{minipage}[t]{3in}
        \begin{center}
        {\Large \bf (#1) $\!\!\!\!\!\!\!\!\!$}
        \epsfig{file=#2,height=2in,width=2in} 
        \end{center}
      \end{minipage}
      \hfill
      \begin{minipage}[t]{3in}
        \begin{center}
        {\Large \bf (#3) $\!\!\!\!\!\!\!\!\!$}
        \epsfig{file=#4,height=2in,width=2in}  
        \end{center}
      \end{minipage}
    \end{figure}}

\newcommand{\threefig}[6]
   {\begin{figure}[h]
      \begin{minipage}[t]{2in}
        \begin{center}
        {\large \bf (#1) $\!\!\!\!\!\!\!\!\!$}
        \epsfig{file=#2,height=1.6in,width=1.6in}
        \end{center}
      \end{minipage}
      \hfill
      \begin{minipage}[t]{2in}
        \begin{center}
        {\large \bf (#3) $\!\!\!\!\!\!\!\!\!$}
        \epsfig{file=#4,height=1.6in,width=1.6in}
        \end{center}
      \end{minipage}
      \hfill
      \begin{minipage}[t]{2in}
        \begin{center}
        {\large \bf (#5) $\!\!\!\!\!\!\!\!\!$}
        \epsfig{file=#6,height=1.6in,width=1.6in}
        \end{center}
      \end{minipage}
    \end{figure}}
         
\newcommand{\twoup}[2]
   {\begin{minipage}{3in}
      \begin{center}
      #1 
      \end{center}
    \end{minipage}
    \hfill
    \begin{minipage}{3in}
      \begin{center}
      #2 
      \end{center}
    \end{minipage}}         

% ******************************************************************
% *** TESTPOINTS SHORTHAND *****************************************

\def\bpz{\begin{problem}{0}}
\newcommand{\bp}[1]{\begin{problem}{#1}}

\def\npz{\newpart{0}}
\newcommand{\np}[1]{\newpart{#1}}

\def\ep{\end{problem}}

%\newcommand{\probpart}...make later
\newcommand{\prob}[1]{\setcounter{problemnum}{#1} \Large \bf \theproblemnum .}


%------------------------------------------------------------------
% PROBLEM, PART, AND POINT COUNTING...

% Create the problem number counter.  Initialize to zero.
\newcounter{problemnum}

% Specify that problems should be labeled with arabic numerals.
\renewcommand{\theproblemnum}{\arabic{problemnum}}


% Create the part-within-a-problem counter, "within" the problem counter.
% This counter resets to zero automatically every time the PROBLEMNUM counter
% is incremented.
\newcounter{partnum}[problemnum]

% Specify that parts should be labeled with lowercase letters.
\renewcommand{\thepartnum}{\alph{partnum}}

% Make a counter to keep track of total points assigned to problems...
\newcounter{totalpoints}

% Make counters to keep track of points for parts...
\newcounter{curprobpts}         % Points assigned for the problem as a whole.
\newcounter{totalparts}         % Total points assigned to the various parts.

% Make a counter to keep track of the number of points on each page...
\newcounter{pagepoints}
% This counter is reset each time a page is printed.

% This "program" keeps track of how many points appear on each page, so that
% the total can be printed on the page itself.  Points are added to the total
% for a page when the PART (not the problem) they are assigned to is specified.
% When a problem without parts appears, the PAGEPOINTS are incremented directly
% from the problem as a whole (CURPROBPTS).


%---------------------------------------------------------------------------


% The \problem environment first checks the information about the previous
% problem.  If no parts appeared (or if they were all assigned zero points,
% then it increments TOTALPOINTS directly from CURPROBPTS, the points assigned
% to the last problem as a whole.  If the last problem did contain parts, it
% checks to make sure that their point values total up to the correct sum.
% It then puts the problem number on the page, along with the points assigned
% to it.

\newenvironment{problem}[1]{
% STATEMENTS TO BE EXECUTED WHEN A NEW PROBLEM IS BEGUN:
%
% Increment the problem number counter, and set the current \ref value to that
% number.
\refstepcounter{problemnum}
%
% Add some vspace to separate from the last problem.
\vspace{0.15in} \par
%
\setcounter{curprobpts}{#1} \setcounter{totalparts}{0}  % Reset counters.
%
% Now put in the "announcement" on the page.
{\Large \bf \theproblemnum. \normalsize ({\it \arabic{curprobpts} point\null\ifnum \value{curprobpts} = 1\else s\fi}\/)}
}{
% STATEMENTS TO BE EXECUTED WHEN AN OLD PROBLEM IS ENDED:
%
% If no parts to problem, then increment TOTALPOINTS and PAGEPOINTS for the
% entire problem at once.
\ifnum \value{totalparts} = 0
        \addtocounter{totalpoints}{\value{curprobpts}}  % Add pts to total.
        \addtocounter{pagepoints}{\value{curprobpts}}   % Add pts to page total.
%
% If there were parts for the problem, then check to make sure they total up
% to the same number of points that the problem is worth. Issue a warning
% if not.
\else \ifnum \value{totalparts} = \value{curprobpts}
        \else \typeout{}
        \typeout{!!!!!!!   POINT ACCOUNTING ERROR   !!!!!!!!}
        \typeout{PROBLEM [\theproblemnum] WAS ALLOCATED \arabic{curprobpts} POINTS,}
        \typeout{BUT CONTAINS PARTS TOTALLING \arabic{totalparts} POINTS!}
        \typeout{}
        \fi
\fi
}


%---------------------------------------------------------------------------


% The \newpart command increments the part counter and displays an appropriate
% lowercase letter to mark the part.  It adds points to the point counter
% immediately.  If 0 points are specified, no point announcement is made.
% Otherwise, the announcement is in scriptsize italics.

\newcommand{\newpart}[1]
{
\refstepcounter{partnum}        % Set the current \ref value to the part number.
\hspace{0.25in}         % Indent the part by a quarter inch.
%
% If points are to be printed for this problem (signaled by point value > 0),
% then put them in in scriptsize italics.
\ifnum #1 > 0
        \makebox[0.5in][l]{{\bf \thepartnum.} {\bf ({\it #1 pt\ifnum #1 = 1\else s\fi\/}) \,\,}}
\else
        \makebox[0.25in][l]{({\bf \thepartnum})}
\fi
%
\hspace{0.1in}          % Lead the material away from the part "number".
%
\addtocounter{totalparts}{#1}   % Add points to totalparts for this problem.
\addtocounter{pagepoints}{#1}   % Add points to total for this page.
\addtocounter{totalpoints}{#1}  % Add points to total for entire test.
}


%---------------------------------------------------------------------------



% Just in case you want to skip some numbers in your test...

\newcommand{\skipproblem}[1]{\addtocounter{problemnum}{#1}}



%---------------------------------------------------------------------------


% The \showpoints command simply gives a count of the total points read in up to
% the location at which the command is placed.  Typically, one places one
% \showpoints command at the end of the latex file, just prior to the
% \end{document} command.  It can appear elsewhere, however.

\newcommand{\showpoints}
{
\typeout{}  
\typeout{====> A TOTAL OF \arabic{totalpoints} POINTS WERE READ.}
\typeout{}
}


%---------------------------------------------------------------------------

\newcommand{\frontmatter}{

% *** COVER *** 

\centerline{\huge \bf \@thetitle}
\begin{center} PHY112 \\ Mesa Community College
\end{center}

\vf \vf

\@thedate \hspace*{1.5in} \bf Name: \hrulefill\\
\hspace*{3.1in}{\mbox{\tiny by writing my name I swear by the honor code}}\\

%Student: Ronald Vermillion \\
%Instructor: Ricardo Ortiz (Tel. 5-0780)
\vf \vf \vf

{\bf Read all of the following information before starting the exam:}
\vs

\begin{itemize}
\item Calculators are allowed.
\item Draw a diagram whenever possible.
\item Show your work.
\item Don't panic.
\item The magnitude of the charge on an electron $\left|e\right|=\unit[1.60\times10^{-19}]{C}$
\item The mass of an electron $m_{e}=\unit[9.11\times10^{-31}]{kg}$
\item The mass of a proton $m_{p}=\unit[1.67\times10^{-27}]{kg}$
\item The permittivity of free space $\varepsilon_{0}=\unit[8.85\times10^{-12}]{\unitfrac{C^{2}}{N \cdot m^{2}}}$.
\item The permeability of free space $\mu_{0}=\unitfrac[1.26\times10^{-6}]{m \cdot kg}{C^{2}}$.

\end{itemize}

\vf \vf \vf

\cpg


}

\oddsidemargin=0in
\evensidemargin=0in
\textwidth=6.5in
\topmargin=-.5in
\textheight=9in
\parindent=0in

\pagestyle{empty}

\makeatother

\makeatother

\usepackage{babel}
\begin{document}

\thetitle{Exam 7: Mini-final}

\thedate{14 July 2016}

\frontmatter

\begin{center}
\emph{This page intentionally left blank.}
\par\end{center}

\newpage{}

\bp{2}
The index of refraction of a type of glass is 1.50,
and the index of refraction of water is 1.33.
If light enters water from this glass,
the angle of refraction will be...\\
\begin{oneparchoices}
\choice equal to the angle of incidence.
\choice greater than the angle of incidence.
\choice less than the angle of incidence.
\end{oneparchoices}
\vfill{}
\ep

\bp{2}
A lighted candle is placed a short distance from a plane mirror,
as shown in the figure.
At which location will the image of the flame appear to be located?\\
\begin{figure}[H]
\centering{}\includegraphics[width=2in]{t7p2}
\end{figure}
\begin{oneparchoices}
\choice at A
\choice at B
\choice at C
\choice at M (at the mirror)
\end{oneparchoices}
\vfill{}
\ep

\bp{2}
A plastic rod is charged up by rubbing a wool cloth,
and brought to an initially neutral metallic sphere
that is insulated from ground.
It is allowed to touch the sphere for a few seconds,
and then is separated from the sphere by a small distance.
After the rod is separated, the rod\\
\begin{oneparchoices}
\choice feels no force due to the sphere.
\choice is attracted to the sphere.
\choice is repelled by the sphere.
\end{oneparchoices}
\vfill{}
\ep

\bp{3}
Two point charges, $Q_1$ and $Q_2$,
are separated by a distance $R$.
If the magnitudes of both charges are doubled
and their separation is also doubled,
what happens to the electrical force 
that each charge exerts on the other one?\\
\begin{oneparchoices}
\choice It increases by a factor of $\sqrt{2}$.
\choice It increases by a factor of $\sqrt{2}$.
\choice It increases by a factor of 4.
\choice It remains the same.
\choice It increases by a factor of 2.
\end{oneparchoices}
\vfill{}
\ep

\bp{3}
How many electrons are necessary to produce $\unit[1.0]{C}$ of negative charge?\\
\begin{oneparchoices}
\choice $1.6\times10^{19}$
\choice $1.6\times10^{9}$
\choice $6.3\times10^{18}$
\choice $6.0\times10^{23}$
\choice $9.0\times10^{9}$
\end{oneparchoices}
\vfill{}
\ep

\newpage{}

\bp{2}
The field inside a charged parallel-plate capacitor is...\\
\begin{oneparchoices}
\choice zero.
\choice uniform.
\choice directed from the negative to the positive plate.
\choice parallel to the plates.
\end{oneparchoices}
\vfill{}
\ep

\bp{3}
You need to construct a $\unit[100]{pF}$ capacitor for a science project.
You plan to cut two $L \times L$ metal squares and place spacers between them.
The thinnest spacers you have are $\unit[0.30]{mm}$ thick.
What is the proper value of $L$?\\
\begin{oneparchoices}
\choice $\unit[2.4]{cm}$
\choice $\unit[2.7]{cm}$
\choice $\unit[3.3]{cm}$
\choice $\unit[5.8]{cm}$
\choice $\unit[34]{cm}$
\end{oneparchoices}
\vfill{}
\ep

\bp{2}
If the electric field between the plates of a given air-filled capacitor
is weakened by removing charge from the plates,
the capacitance of that capacitor\\
\begin{oneparchoices}
\choice increases.
\choice decreases.
\choice does not change.
\choice It cannot be determined from the information given.
\end{oneparchoices}
\vfill{}
\ep

\bp{3}
The length of a certain wire is kept the same while its radius is doubled.
What is the new resistance of this wire?\\
\begin{oneparchoices}
\choice It is decreased by a factor of 8.
\choice It is decreased by a factor of 4.
\choice It is decreased by a factor of 2.
\choice It is increased by a factor of 2.
\choice It is increased by a factor of 4.
\end{oneparchoices}
\vfill{}
\ep

\bp{2}
How much charge must pass by a point in a wire in $\unit[10]{s}$
for the current in the wire to be $\unit[0.5]{A}$?\\
\begin{oneparchoices}
\choice $\unit[2.0]{C}$
\choice $\unit[0.050]{C}$
\choice $\unit[20]{C}$
\choice $\unit[5.0]{C}$
\end{oneparchoices}
\vfill{}
\ep

\bp{2}
Which, if any, of these statements are true? 
(More than one may be true.)\\
\begin{oneparchoices}
\choice A battery supplies energy to a circuit.
\choice A battery is a source of current. The current leaving the battery is always the same.
\choice A battery is a source of potential difference. The potential difference between the terminals of the battery is always the same.
\end{oneparchoices}
\vfill{}
\ep

\newpage{}

\bp{2}
You have a collection of $\unit[1.0]{k\Omega}$ resistors.
How can you connect four of them
to produce an equivalent resistance of $\unit[0.25]{k\Omega}$?\\
\begin{oneparchoices}
\choice (Two in parallel) in series with (two in parallel).
\choice (Two in series) in parallel with (two in series).
\choice All in parallel.
\choice All in series.
\end{oneparchoices}
\vfill{}
\ep

\bp{3}
A $\unit[4.0]{\Omega}$ resistor is connected to a $\unit[12]{\Omega}$ resistor
and this combination is connected to an ideal dc power supply
with voltage $V$ as shown in the figure.
If the total current in this circuit is $I = \unit[2.0]{A}$,
what is the value of voltage $V$?\\
\begin{figure}[H]
\centering{}\includegraphics[width=2in]{t7p15}
\end{figure}
\begin{oneparchoices}
\choice $\unit[2.0]{V}$
\choice $\unit[6.0]{V}$
\choice $\unit[8.0]{V}$
\choice $\unit[3.0]{V}$
\choice $\unit[1.5]{V}$
\end{oneparchoices}
\vfill{}
\ep

\bp{3}
A uniform $\unit[2.5]{T}$ magnetic field points to the right.
A $\unit[3.0]{m}$-long wire, carrying $\unit[15]{A}$,
is placed at an angle of $30^\circ$ to the field,
as shown in the figure.
What is the magnitude of the force on the wire?\\
\begin{figure}[H]
\centering{}\includegraphics[width=2in]{t7p18}
\end{figure}
\begin{oneparchoices}
\choice $\unit[19]{N}$
\choice $\unit[56]{N}$
\choice $\unit[65]{N}$
\choice $\unit[80]{N}$
\choice $\unit[113]{N}$
\end{oneparchoices}
\vfill{}
\ep

\bp{2}
Refer to the situation in the previous problem.
What is the direction of the force on the wire?\\
\begin{oneparchoices}
\choice left
\choice right
\choice into the page
\choice out of the page
\end{oneparchoices}
\vfill{}
\ep

\newpage{}

\bp{2}
A flat coil is in a uniform magnetic field.
The magnetic flux through the coil is greatest when the plane of its area is\\
\begin{oneparchoices}
\choice parallel to the magnetic field.
\choice perpendicular to the magnetic field.
\choice at $45^\circ$ with the magnetic field.
\end{oneparchoices}
\vfill{}
\ep

\bp{2}
A coil lies flat on a level tabletop in a region 
where the magnetic field vector points straight up.
The magnetic field suddenly grows stronger.
When viewed from above, 
what is the direction of the induced current
in this coil as the field increases?\\
\begin{oneparchoices}
\choice counterclockwise
\choice clockwise
\choice clockwise initially, then counterclockwise before stopping
\choice There is no induced current in this coil. 
\end{oneparchoices}
\vfill{}
\ep

\bp{0}
What does the fox say?\\
\begin{oneparchoices}
\choice ``Baby don't hurt me. Don't hurt me no more.''
\choice ``For whom the bell tolls.''
\choice ``What can I do for my country?''
\choice ``All your base are belong to us.''
\choice ``Fraka-kaka-kaka-kaka-kow!''
\end{oneparchoices}
\vfill{}
\ep

\end{document}
