\documentclass{exam}

\oddsidemargin=0in
\evensidemargin=0in
\textwidth=6.5in
\topmargin=-.5in
\textheight=9in
\parindent=0in
\pagestyle{empty}

\begin{document}
\begin{center}
  Test 3: Introduction to Electronics\\

  PHY101 - Fall 2016\\
\end{center}
Name:\\

\begin{questions}

\question An $X$ source in electronics is one which attempts to maintain constant $X$, regardless of the circuit to which it is connected.
\begin{choices}
  \choice A battery is a voltage source.
  \choice A battery is a current source.
  \choice A battery is a power source.
  \choice A battery is none of the above.
\end{choices}
\vfill{}

\question Choose the best statement:
\begin{choices}
  \choice Current increases when resistors are connected in parallel because there are more paths to flow around.
  \choice Current decreases when resistors are connected in parallel because there are more paths to flow around.
  \choice Current increases when resistors are connected in parallel because the equivalent resistance is the sum of the individual resistances.
  \choice Current decreases when resistors are connected in parallel because the equivalent resistance is the sum of the individual resistances.
\end{choices}
\vfill{}

\question How does the resistance of a light bulb change as the light bulb heats up?
\begin{choices}
  \choice Its resistance increases. 
  \choice Its resistance decreases.
  \choice Its resistance remains constant.
  \choice Its resistance starts high but decreases once the applied voltage exceeds approximately 0.6 V.
\end{choices}
\vfill{}

\question What was the order of magnitude of the resistance of the light bulbs we used in lab?
\begin{choices}
  \choice 1 m$\Omega$
  \choice 1 $\Omega$
  \choice 1 k$\Omega$
  \choice 1 M$\Omega$
\end{choices}
\vfill{}

\question How does resistance of a wire change as its dimensions change?
\begin{choices}
  \choice Resistance is directly proportional to cross-sectional area and directly proportional to length. ($R \propto A\,L$)
  \choice Resistance is directly proportional to cross-sectional area and inversely proportional to length. ($R \propto \frac{A}{L}$)
  \choice Resistance is inversely proportional to cross-sectional area and directly proportional to length. ($R \propto \frac{L}{A}$)
  \choice Resistance is inversely proportional to cross-sectional area and inversely proportional to length. ($R \propto \frac{1}{L\,A}$)
\end{choices}
\vfill{}

\newpage{}

\question Define series.
\vfill{}

\question Define parallel.
\vfill{}

\question Define voltage.
\vfill{}

\question Define current.
\vfill{}

\question Define resistance (use Ohm's law!).
\vfill{}

\question Give one example of an ohmic material and one example of a non-ohmic material.
\vfill{}


\end{questions}

\end{document}
