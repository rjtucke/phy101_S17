\documentclass{article}
\oddsidemargin=0in
\evensidemargin=0in
\textwidth=6.5in
\topmargin=-.5in
\textheight=9in
\parindent=0in
\pagestyle{empty}

\begin{document}
Batteries and bulbs activity
\begin{itemize}
  \item Obtain a single battery, bulb, and wire. Connect them in as many ways as you can. Sketch at least two arrangements that cause the bulb to light up and at least two that don't.
  \item Summarize how a battery and bulb must be connected to light the bulb.
  \item Predict, then test which bulb will be brighter: (DIAGRAM 1: Battery with a bulb vs two batteries in series with one bulb.) Was your prediction correct?
\end{itemize}

Current is the word for whatever it is that flows through an electric circuit.
An ammeter measures current. IMPORTANT: THE AMMETER MUST \textbf{NEVER} BE CONNECTED DIRECTLY ACROSS THE BATTERY OR IT WILL BE DAMAGED. IT MUST ALWAYS BE IN SERIES WITH A ``RESISTANCE''.

\begin{itemize}
  \item Hook up an ammeter in line with each of the circuits from the previous part. How does the reading on the ammeter compare to the brightness of the bulb?
  \item Move the ammeter ``around'' the circuit: how do the readings compare? (Do \textbf{not} connect it in parallel with the battery!) Be sure that you measure the current ``before'' the bulb and ``after'' the bulb. How much of the current that enters a bulb then leaves it? Is current ``used up'' after it passes through a bulb?
  \item Predict, then test the relationship of brightness between the three bulbs: (DIAGRAM 2: Two batteries in series with one bulb vs two batteries and two bulbs, all in series.) Was your prediction correct?
  \item How does current change when bulbs are added in series?
  \item A bulb can be thought of as an obstacle or resistance to current. How does the \textbf{total} resistance in a circuit change when bulbs are added in series?
  \item Predict, then test the brightness of each bulb: DIAGRAM 3: (Battery with one bulb vs battery with two bulbs in parallel.) Was your prediction correct?
  \item Predict, then test how the current through battery 1 compares to the current through battery 2. Was your prediction correct? Compare this to the previous case when two batteries were in series.
  \item Use an ammeter to measure the current quantitatively in the second circuit, above. What is the relationship between the current leaving the battery and the current through each bulb?
  \item Finally, compare the current through battery 1 with battery 2: Which circuit has the greater opposition (resistance) to current? How is this case different from when the two bulbs were in series?
\end{itemize}

\end{document}
