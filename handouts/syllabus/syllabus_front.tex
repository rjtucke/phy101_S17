\documentclass{report}
%\usepackage{ae,aecompl}
%\usepackage[T1]{fontenc}
%\usepackage[latin1]{inputenc}
\usepackage{geometry}
\geometry{verbose,tmargin=1.25in,bmargin=1.25in,lmargin=1.25in,rmargin=1.25in}
\usepackage{array}

\makeatletter

%%%%%%%%%%%%%%%%%%%%%%%%%%%%%% LyX specific LaTeX commands.
\newcommand{\noun}[1]{\textsc{#1}}
%% Because html converters don't know tabularnewline
\providecommand{\tabularnewline}{\\}

%%%%%%%%%%%%%%%%%%%%%%%%%%%%%% User specified LaTeX commands.
\usepackage{framed}

\makeatother

%\usepackage{babel}
\begin{document}

\section*{Physics 101 -- Introduction to Physics}

\section*{Arizona State University}

\begin{center}
\begin{tabular}{ll}
Fall 2016                     & Ross Tucker (ross.tucker@asu.edu) \\
Section 70355                 & \textbf{Mastering Physics ID:} TUCKER70355 \\
\textbf{Class times:} (CLCC 254) MW:~4:35--5:50 & \textbf{Office hours:} (CLCC 217C) W:~2--4 \\
\textbf{Labs:} (CLCC 385)     & \\
M:~10:30--1:15 (73103)        & M:~1:30--4:15 (73102) \\

\end{tabular}
\par\end{center}

\section*{Course Description}

Emphasizes applications of physics to life in the modern world.
Presumes understanding of elementary algebra.

\section*{Lecture Grading}
The grading scale is a simple linear scale: $\ge$90\% is an A and
so on. Weights are as follows.

\subsection*{Homework (35\% of grade)}
Most homework is assigned through the online Mastering Physics system.
Additionally, a single written homework question (due at the beginning of the
next class) will be given at the end of most lectures.

\subsection*{Participation and Quizzes (15\% of grade)}
Discussion is an important part of this class. Active participation is required.
Keeping up with assigned reading is vital, so there will be a short quiz at the 
beginning of most lectures.

\subsection*{Tests (30\% of grade)}
Three in-class tests will be given during the semester.

\subsection*{Final exam (20\% of grade)}
The comprehensive final exam will be held on December 5.

\newpage{}

\section*{Laboratory Grading}
Lab activities are important to get hands-on experience with physical principles
and phenomena discussed in class. You do not need to purchase a lab manual. Instead,
lab procedures will be posted on Blackboard for you to print out and bring with you.
As with lecture, the grading scale is a simple linear scale. Weights are as follows.

\subsection*{Prelab quiz (20\% of grade)}
To prepare for in-lab work, you must read the lab procedure in advance and
review the relevent background material in the main course textbook. You
will answer prelab questions and turn in answers at the beginning of each
section. There is no prelab quiz for the first lab meeting.

\subsection*{Quizzes (20\% of grade)}
Each section will have a quiz at the end with questions regarding the current
activity AND the previous completed activity.

\subsection*{Participation (20\% of grade)}
Complettion of every activity is required. There are no make-up sessions for
missed labs. Failure to complete a lab activity will result in a lower final 
grade.

\subsection*{Short report (20\% of grade)}
At the end of each lab session, each group will submit a short lab report.

\subsection*{Formal report (20\% of grade)}
For one or two selected activities (to be announced during the semester), an individual
report is required.

\end{document}
