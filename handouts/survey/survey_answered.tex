%% LyX 2.1.3 created this file.  For more info, see http://www.lyx.org/.
%% Do not edit unless you really know what you are doing.
\documentclass[american]{article}
\usepackage{ae,aecompl}
\usepackage[T1]{fontenc}
\usepackage[latin1]{inputenc}
\usepackage{geometry}
\geometry{verbose,tmargin=1.25in,bmargin=1.25in,lmargin=1.25in,rmargin=1.25in}
\usepackage{array}

\usepackage{babel}
\begin{document}

Ross Tucker

\textbf{Why do you want to teach/do physics?}\\
I am a physicist
because there is nothing else so fascinating
or so deserving of my time and efforts. 
I teach physics
because I can't help but share the things which interest me.

\textbf{What do you think of when you think of physics? 
What do you think of when you think of physicists?}\\
When I think of physics, I think first of my own branch,
particle physics.
Particle physics is the search for the fundamental constitutents of the universe
and the rules that govern them. Particle physicists use enormous accelerators
to produce exotic types of matter and energy in an attempt to understand the
truly basic rules that underlie the entire cosmos.
I also think of the many other fields of physics,
many of which are interdisciplinary,
which apply similar techniques to VERY different phenomena
often with amazing success.

When I think of physicists,
I think of my colleagues, 
almost none of whom look or act like a stereotypical scientist.
As a discipline, Physics still suffers from sexism, homophobia, 
and racism, but that is changing,
particularly in the current generation of young scientists.
For example, more than half of the physicists my age in my working research group 
are women.
Directly opposite to the stereotype of a scientist being interested only in
science, I've found that scientists tend to be interested in MANY different things.
For some reason, a great number of physicists also tend to be half-way decent musicians, too.

\textbf{What do you hope students will get out of this class?}\\
I hope that students in this class will take away
a good understanding of the fundamental ideas in mechanics, 
electricity and magnetism, optics, and
some of the discoveries in modern physics.
I also hope that students will appreciate that physics is not a collection 
of facts but rather a process by which people generate new knowledge about
how the world works.
Finally, I hope that students in this class will see through the myth that
you have to be a genius to understand or enjoy science and that physics is
not an exclusive priesthood, accessible only to a privileged few.

\textbf{Have you ever taught a physics class before? What was it like?}\\
I've been teaching for most of my life.
 I created an educational outreach program about space exploration at age 12
which I presented to more than 5000 people from California to Texas to classrooms
and at the Arizona Science Center. As an undergrad, I enjoyed tutoring and briefly
attempted to make a career of it before applying for graduate school. As a
graduate student, I was the TA for 15 sections of freshman physics courses. Before
coming to ASU as an instructor, I taught night classes at Mesa Community College
for $2~\frac{1}{2}$ years. The one and only generalization I can make about all of
these experiences is that every time it has been different and every time it has
been thrilling.
\end{document}
