\documentclass{article}
\oddsidemargin=0in
\evensidemargin=0in
\textwidth=6.75in
\parindent=0in
\renewcommand{\baselinestretch}{1.5}
\pagestyle{empty}
\begin{document}
\begin{center}
  \Huge{Intermediate electronics notes}
\end{center}
\begin{itemize}

\item \textbf{Wave terminology}
  \begin{itemize}
    \item \textbf{Amplitude} measures the `height' of the wave, the difference
          between its minumum and maximum values. For example, the amplitude 
          of a sound wave is related to its volume. Amplitude is usually 
          written with the symbol and units of the quantity which is `waving' 
          but a general wave amplitude is written $A$.
    \item \textbf{Period} measures the time between similar points on the same
          wave. Period is usually written $T$ and measured in seconds.
    \item \textbf{Frequency} measures how many wiggles happen each second.
          Frequency is usually written $f$ and measured in Hertz (Hz), which
          is shorthand for inverse seconds ($\frac{1}{s}$).
    \item It is often convenient to define \textbf{angular frequency}, $\omega$
          (omega), measured in radians per second. Angular frequency is
          related to regular frequency the equation $\omega = 2\pi\,f$.
    \item Frequency is related to period by the equation $f=\frac{1}{T}$. 
    \item When considering two waves of the same frequency, the 
          \textbf{phase shift} measures the delay between similar points,
          Phase shift is usually written $\phi$ (phi) and measured in degrees.
          For example, if the time between two wave peaks is 1 second and the 
          waves have a period of 4 seconds, then the phase shift is 90 degrees.
    \item As a moving wave passes through space, its frequency and wave speed
          cause periodic patterns in space. The distance between similar points
          at a given moment of time is the \textbf{wavelength} usually written
          $\lambda$ (lambda) and measured in meters.
    \item Frequency, wave speed, and wavelength are related by the equation
          $v = f\,\lambda$.
    \item \textbf{Timbre} is `quality' of a sound, governed by the distribution
          of frequencies in the (generally) complex spectrum. It only applies
          to sound.
    \item Hewitt discusses amplitude, period, frequency, wavelength on pages 
          358-359. Timbre is discussed on page 395. An example of phase shift
          in light is on page 552. The frequency-wavelength equation is 
          discussed on page 362.
  \end{itemize}

\newpage{}

\item \textbf{The Fourier Transform}
  \begin{itemize}
    \item In lab, we saw that each sound can be represented two ways:
          amplitude vs time or amplitude vs frequency (the spectrum). The 
          computer used an algorithm called the `Fast Fourier Transform' (FFT) 
          to show us the spectrum for many different sounds.
    \item On a graph of amplitude vs frequency, a low pitch tone appears on the
          left whereas a high pitch tone appears on the right. Louder sounds
          have greater height (like normal) and softer sounds are smaller. It
          is much easier to compare timbres on this kind of graph: more 
          complicated graphs sound `rich' and simpler (line-like) graphs 
          sound `pure'.
    \item The spectrum of a square wave follows a mathematically simple 
          pattern. In lab, we generated a square wave with the signal 
          generator and viewed its spectrum using the FFT tool. We 
          discovered that a square wave is composed of discrete sine waves.
          The individual frequencies are odd-number multiples of the 
          square wave's frequency. For example, a 1000 Hz square wave has
          a spectrum with peaks at 1000 Hz, 3000 Hz, 5000 Hz, and so on.
          The amplitudes of the individual waves also follow a simple pattern: 
          the 3000 Hz wave has amplitude $\frac{1}{3}$, the 5000 Hz wave has 
          amplitude $\frac{1}{5}$ and so on. 
          (Please remember that this is ONLY true for a square wave- other 
          wave shapes will have different patterns.)
    \item Hewitt discusses the Fourier transform on page 397-399. Additional 
          examples of spectra are on page 489, 566, 570, and 606.
  \end{itemize}

\item \textbf{Response of a series RC circuit to AC}
  \begin{itemize}
    \item When an AC voltage is applied to a series RC circuit, the capacitor
          voltage `lags' that of the source voltage.
    \item The degree to which the capacitor opposes the current flow depends
          on the frequency of the applied voltage and the capacitance of the
          capacitor. At higher frequencies, the capacitor is `less important'.
          At lower frequencies, the capacitor blocks more of the signal.
          Intuitively, we can explain this result by saying that higher
          frequencies give the capacitor less time to accumulate charge before
          the polarity reverses- since less charge accumulates, there is less 
          voltage on the capacitor and more for the resistor. On the other hand,
          at low frequency, the charge on the capacitor can `keep up' with the
          source voltage and therefore current is kept low.
  \end{itemize}

\newpage{}

\item \textbf{Magnetism}
  \begin{itemize}
    \item Permanent magnets has been known of since prehistory.
          Electromagnetism was discovered by Oersted in 1820.
          In fact, electromagnetism is simpler to understand so we start there.
    \item \textbf{Magnetism results from current flow.} Whenever current flows,
          a magnetic field is created.
    \item When current changes, energy is stored or released from its magnetic
          field. Effectively, currents have a sort of \textit{magnetic inertia} 
          (not related to ordinary momentum)- if you try to `turn off' a 
          current suddenly, the collapse of the magnetic field keeps the 
          current going and if you try to increase a current suddenly, the 
          magnetic field absorbs the initial energy until the current can 
          increase smoothly. Formally, \textbf{a changing magnetic field induces
          an electric field}.
    \item In lab, we changed the current flowing through a wire on an iron core
          and measured the induced current in a nearby wire. This phenomenon
          is the basis for the circuit components called inductors and 
          transformers.
    \item Hewitt discusses magnetism on pages 453-460 and induction on 470-480.
  \end{itemize}

\item \textbf{Extending Ohm's Law}
  \begin{itemize}
    \item When current flows through a resistor, electrical energy is 
          converted to heat and lost to the circuit. Moreover, the voltage 
          across a resistor is directly proportional to the current through it.
    \item Capacitors have a resistor-like behavior, in that the voltage 
          across a capacitor is directly proportional to the current through it 
          (though voltage and current are out of phase). Energy in a capacitor 
          is not dissipated. Instead, it is stored in the electric field 
          between the capacitor plates.
    \item Inductors have a similar behavior- energy is stored in the magnetic 
          field, resulting in a directly proportional relationship between 
          applied voltage and current. Voltage and current are also out of phase
          in an inductor.
    \item Since energy is not lost in capacitors and inductors, the ratio of
          voltage to current is not a resistance. We define a new term,
          \textbf{reactance} for this idea. The reactance of a capacitor is
          $X_{C} = \frac{1}{2\pi\,f\,C}$ where $f$ is the frequency of the AC
          and $C$ is its capacitance. The reactance of an inductor is 
          $X_{L} = 2\pi\,f\,L$ where $L$ is its inductance.
    \item Resistance and reactance can be combined into an \textbf{impedance}.
          The ratio of the magnitude of voltage to magnitude of current is
          $Z = \sqrt{R^{2} + (X_{L} - X_{C})^{2}}$ and the phase shift is
          $\phi = \arctan \left ( \frac{X_{L}-X_{C}}{R} \right )$.
  \end{itemize}

\end{itemize}
\end{document}
