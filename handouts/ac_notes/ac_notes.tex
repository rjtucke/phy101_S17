\documentclass{article}
\oddsidemargin=0in
\evensidemargin=0in
\textwidth=6.75in
\parindent=0in
\renewcommand{\baselinestretch}{1.5}
\pagestyle{empty}
\begin{document}
\begin{center}
  \Large{Intermediate electronics notes}
\end{center}
We began this lab with a goal of understanding waves in general.  Electricity, 
sound, light, mechanical vibrations, and other phenomena feature wave behavior 
and we can apply similar terminology and techniques to analyze them.
\begin{itemize}

\item \textbf{Wave terminology}
  \begin{itemize}
    \item \textbf{Amplitude} measures the `height' of the wave, the difference
          between its minumum and maximum values. For example, the amplitude 
          of a sound wave is related to its volume. Amplitude is usually 
          written with the symbol and units of the quantity which is `waving' 
          but a general wave amplitude is written $A$.
    \item \textbf{Period} measures the time between similar points on the same
          wave. Period is usually written $T$ and measured in seconds.
    \item \textbf{Frequency} measures how many wiggles happen each second.
          Frequency is usually written $f$ and measured in Hertz (Hz), which
          is shorthand for inverse seconds ($\frac{1}{s}$).
    \item It is often convenient to define \textbf{angular frequency}, $\omega$
          (omega), measured in radians per second. Angular frequency is
          related to regular frequency the equation $\omega = 2\pi\,f$.
    \item Frequency is related to period by the equation $f=\frac{1}{T}$. 
    \item When considering two waves of the same frequency, the 
          \textbf{phase shift} measures the delay between similar points,
          Phase shift is usually written $\phi$ (phi) and measured in degrees.
          For example, if the time between two wave peaks is 1 second and the 
          waves have a period of 4 seconds, then the phase shift is 90 degrees.
    \item As a moving wave passes through space, its frequency and wave speed
          cause periodic patterns in space. The distance between similar points
          at a given moment of time is the \textbf{wavelength} usually written
          $\lambda$ (lambda) and measured in meters.
    \item Frequency, wave speed, and wavelength are related by the equation
          $v = f\,\lambda$.
    \item \textbf{Timbre}
  \end{itemize}

\item \textbf{The Fast Fourier Transform (FFT)}
  \begin{itemize}
    \item 
  \end{itemize}

\item \textbf{Fourier series for square wave}
  \begin{itemize}
    \item 
  \end{itemize}

\item \textbf{Response of a series RC circuit to AC}
  \begin{itemize}
    \item When an AC voltage is applied to a series RC circuit, the capacitor
          voltage `lags' that of the source voltage, passing some current.
    \item The degree to which the capacitor opposes the current flow depends
          on the frequency of the applied voltage and the capacitance of the
          capacitor.
    \item
  \end{itemize}

\item \textbf{Magnetism}
  \begin{itemize}
    \item Permanent magnets has been known of since prehistory.
          Electromagnetism was discovered by Oersted in 1820.
          In fact, electromagnetism is simpler to understand so we start there.
    \item \textbf{Magnetism results from current flow.} Whenever current flows,
          a magnetic field is created.
  \end{itemize}

\item \textbf{Extending Ohm's Law}
  \begin{itemize}
    \item When current flows through a resistor, electrical energy is 
          converted to heat and lost to the circuit. Moreover, the voltage 
          across a resistor is directly proportional to the current through it.
    \item Capacitors have a resistor-like behavior, in that the voltage 
          across a capacitor is directly proportional to the current through it 
          (though voltage and current are out of phase). Energy in a capacitor 
          is not dissipated. Instead, it is stored in the electric field 
          between the capacitor plates.
    \item Inductors have a similar behavior- energy is stored in the magnetic 
          field, resulting in a directly proportional relationship between 
          applied voltage and current. Voltage and current are also out of phase
          in an inductor.
    \item Since energy is not lost in capacitors and inductors, the ratio of
          voltage to current is not a resistance. We define a new term,
          \textbf{reactance} for this idea.
    \item The equation for \textbf{capacitive reactance} is 
          $X_{C} = \frac{1}{2\pi\,f\,C}$ where $f$ is the frequency of the AC
          and $C$ is the capacitance of the capacitor.
    \item The equation for \textbf{inductive reactance} is 
          $X_{L} = 2\pi\,f\,L$ where $f$ is the frequency of the AC and $L$ is
          the inductance of the inductor.
    \item Resistance and reactance can be combined into an \textbf{impedance}.
          The ratio of the magnitude of voltage to magnitude of current is
          $Z = \sqrt{R^{2} + (X_{L} - X_{C})^{2}}$ and the phase shift is
          $\phi = \arctan \left ( \frac{X_{L}-X_{C}}{R} \right )$.
  \end{itemize}

\end{itemize}
\end{document}
