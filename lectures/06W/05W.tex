\documentclass[english]{beamer}
\usepackage[per-mode=fraction]{siunitx}
\useoutertheme{RJTcustom}

\begin{document}

\begin{frame}{Lecture 05W Plan}
  \textbf{Material}
  \begin{itemize}
    \item Energy
    \item Work and Potential Energy
    \item Energy vs Momentum
    \item NTQ for Monday
  \end{itemize}
\end{frame}

\begin{frame}{Energy}
  Kinds of energy:
  \begin{itemize}
    \item Kinetic (motion) $KE=\frac{1}{2} m\,v^{2}$
    \item Potential: When a force depends only on position
    \item Other: Electrical, thermal (heat), chemical, nuclear
  \end{itemize}
\end{frame}

\begin{frame}{Work and Potential Energy}
  \begin{itemize}
    \item When an external force changes the energy of a system, is does \emph{work}.
    \item In general, $W=F\,\Delta x$.
    \item To find any potential energy equation, simply calculate the work done to get there:
    \item Example: Gravitational potential energy is $\text{GPE} = m\,g\,y$.
    \item Note that all potential energy equations depend on where `getting there' is considered to start from! 
  \end{itemize}
\end{frame}

\begin{frame}{Energy vs Momentum}
  Energy and momentum share some similarities.
  \begin{itemize}
    \item Kinetic energy and momentum both quantify motion.
    \item Unless acted on by an external force, both energy and momentum of a system are conserved.
    \item An external force changes the energy of a system via ``work'' just as it changes momentum via ``impulse''.
  \end{itemize}
  However, they are different things.
  \begin{itemize}
    \item Energy is a scalar. Momentum is a vector.
    \item Energy can be ``stored'' as potential or internal energy. Momentum is always visible.
  \end{itemize}
\end{frame}

\begin{frame}{NTQ for Monday}
  A cart with a mass of $\SI{0.5}{\kilogram}$ on a frictionless track moves at $\SI{1.0}{\meter\per\second}$ to the right. It strikes a $\SI{0.8}{\kilogram}$ cart, which was moving at $\SI{1.2}{\meter\per\second}$ to the left. What is the magnitude of the system's kinetic energy?
\end{frame}

\end{document}
