\documentclass[english]{beamer}
\usepackage[per-mode=fraction]{siunitx}
\useoutertheme{RJTcustom}

\begin{document}

\begin{frame}{Lecture 06W Plan}
  \textbf{Material}
  \begin{itemize}
    \item Historical Note
    \item Projectile Motion
    \item Orbital Motion
    \item Universal Law of Gravitation
    \item Dark Matter
    \item NTQ for Monday
  \end{itemize}
\end{frame}

\begin{frame}{Hysterical Note}
  Johannes Kepler correctly described orbital motion in 1609.
\end{frame}

\begin{frame}{Projectile Motion}
  \begin{itemize}
    \item The path of a projectile in free-fall is a parabola.
    \item In the vertical direction, the motion has constant acceleration.
    \item In the horizontal direction, the motion is uniform.
  \end{itemize}
\end{frame}

\begin{frame}{Orbital Motion}
  \begin{itemize}
    \item If a projectile moves far enough that the curvature of the earth is important,
          it turns out that the parabola is actually part of an ellipse.
    \item If the ellipse is big enough, then the object ``falls''  all the way around the planet.
    \item This is orbital motion: falling with enough sideways motion to ``miss'' the ground.
  \end{itemize}
\end{frame}


\begin{frame}{Universal Law of Gravitation}
  \begin{itemize}
    \item The gravitational field $g$ (in $W=-m\,g$) is not a universal constant.
    \item The gravitational field can be calculated using the equation $g=\frac{G\,M}{R^{2}}$, where \\
    \begin{itemize}
          \item $G=\SI{6.67e-11}{\meter^3\per\second^2\kilogram}$ is Newton's constant,
          \item $M$ is the mass of the gravitating object, and
          \item $R$ is the distance to the center of the gravitating object.
    \end{itemize}
  \end{itemize}
\end{frame}

\begin{frame}{Universal Law of Gravitation}
  \begin{itemize}
    \item For example, $R_{\text{earth}}=\SI{6.35e6}{\meter}$ and $M_{\text{earth}}=\SI{5.97e24}{\kilogram}$ so $g=\frac{\left(6.67\times10^{-11}\right) \times \left(5.97\times10^{24}\right)}{\left(6.35\times10^6\right)^2}=\SI{9.8}{\meter\per\second^2}$ at the surface.
    \item Group exercise: Calculate the gravitational field at the International Space Station (altitude = $\SI{400}{\km}$).
  \end{itemize}
\end{frame}

\begin{frame}{Universal Law of Gravitation}
  The space station's altitude is about $\SI{400}{\km}$ ($R=\SI{6.75e6}{\meter}$) so $g=\SI{8.7}{\meter\per\second^2}$.
\end{frame}

\begin{frame}{Universal Law of Gravitation}
  \begin{itemize}
    \item The acceleration of circular motion is given by the equation $a=\frac{v^2}{R}$.
    \item If this is due to gravity, $F=\frac{G\,M\,m}{R^2}$ can be used to find for mass of the orbited body.
    \item This is how we measure the mass of the Sun, Earth, etc: observe the radius and period of something orbiting it.
  \end{itemize}
\end{frame}

\begin{frame}{Dark Matter}
  \begin{itemize}
    \item When we use this technique to measure the mass of the galaxy, we get a number which cannot be reconciled with the visible stars and dust.
    \item There must exist a different kind of matter than any we have seen.
    \item We call it \emph{dark matter}.
    \item This is one of the ``known unknowns'' of modern physics.
  \end{itemize}
\end{frame}

\begin{frame}{NTQ for Monday}
  Since the Moon is gravitationally attracted to Earth, why doesn't it simply crash into Earth.
\end{frame}

\end{document}
