\documentclass[english]{beamer}
\usepackage[per-mode=fraction]{siunitx}
\useoutertheme{RJTcustom}

\begin{document}

\begin{frame}{Lecture 13W Plan}
  \textbf{Material}
  \begin{itemize}
    \item Historical Note: Balmer
    \item Physics Bingo
    \item Particle Physics Slides
    \item Discussion
  \end{itemize}
\end{frame}

\begin{frame}{Historical Note: Balmer}
  In 1885, Johann Balmer discovered patterns in the spectrum of glowing hydrogen gas. This discovery paved the way to explore the structure of an atom by studying its spectrum.
\end{frame}

\begin{frame}{Physics Bingo}
  Keep an eye out for the following familiar faces:\\
  \begin{itemize}
    \item Newton's First Law: Inertia.
    \item Newton's Second Law: Force.
    \item Electrical and magnetic fields.
    \item Waves in a medium propagate at a fixed speed.
    \item Spectral analysis (like the FFT).
  \end{itemize}
  Make a note when you see each one.
\end{frame}

\begin{frame}{Particle Physics Slides}
  
\end{frame}

\begin{frame}{Discussion}
  What familiar faces did you see in the presentation?\\
  \begin{itemize}
    \item Newton's First Law: Inertia.
    \item Newton's Second Law: Force.
    \item Electrical and magnetic fields.
    \item Waves in a medium propagate at a fixed speed.
    \item Spectral analysis (like the FFT).
  \end{itemize}
\end{frame}

\begin{frame}{Written homework for Monday}
  \begin{itemize}
    \item Pick one of the `familiar faces' you just noted.
    \item Write a \textit{short} paragraph which includes:
    \begin{itemize}
      \item A description of the physics principle.
      \item How that principle is used at Jefferson Lab.
    \end{itemize}
    \item Due Monday at the beginning of lecture.
  \end{itemize}
\end{frame}

\end{document}
