\documentclass{exam}

\oddsidemargin=0in
\evensidemargin=0in
\textwidth=6.5in
\topmargin=-.5in
\textheight=9in
\parindent=0in
\pagestyle{empty}

\begin{document}
\begin{center}
  Final Exam\\

  PHY101 - Fall 2016\\
\end{center}
Name:\\

\begin{questions}

\question If a car is moving at 90 km/h and it rounds a corner, also at 90 km/h, does it maintain a constant speed? A constant velocity?
\begin{choices}
  \choice It maintains a constant velocity, but not a constant speed.
  \choice It maintains a constant speed, but does not maintain a constant velocity.
  \choice It maintains a constant speed and a constant velocity.
  \choice It does not maintain a constant speed or a constant velocity.
\end{choices}
\vfill{}

\question Acceleration is generally defined as the time rate of change of velocity. When can it be defined as the time rate of change of speed?
\begin{choices}
  \choice When moving in a straight line, the time rate of change of speed is acceleration.
  \choice When moving in a circle, the time rate of change of speed is always zero.
  \choice When moving in a straight line, the time rate of change of speed is called velocity.
  \choice When moving in a circle, the time rate of change of speed is called average speed.
\end{choices}
\vfill{}

\question What is the acceleration of a car that maintains a constant velocity of 10 km/h for 10 s?
\begin{choices}
  \choice 1 km/h·s
  \choice 10 km/h·s
  \choice 0 km/h·s
  \choice 0.1 km/h·s
\end{choices}
\vfill{}

\question A freely-falling student falls with constant
\begin{choices}
  \choice velocity.
  \choice distances each successive second.
  \choice acceleration.
  \choice speed.
\end{choices}
\vfill{}

\question A professor undergoes acceleration when she
\begin{choices}
  \choice loses speed.
  \choice gains speed.
  \choice changes her direction.
  \choice all of the above
\end{choices}
\vfill{}

\newpage{}

\question If no external forces act on a moving object, it will
\begin{choices}
  \choice come to an abrupt halt.
  \choice move slower and slower until it finally stops.
  \choice continue moving at the same speed.
  \choice none of the above
\end{choices}
\vfill{}

\question How does acceleration depend on the net force?
\begin{choices}
  \choice Acceleration is equal to the net force.
  \choice Acceleration is independent of the net force.
  \choice Acceleration is inversely proportional to the net force.
  \choice Acceleration is proportional to the net force.
\end{choices}
\vfill{}

\question If you push on a crate with a horizontal force of 100 N and it slides at constant velocity, what is the magnitude and direction of the frictional force acting on the crate?
\begin{choices}
  \choice The frictional force on the crate is 100 N opposite the direction of motion.
  \choice The frictional force on the crate is 100 N in the same direction as the direction of motion.
  \choice The frictional force has nothing to do with you pushing on the crate.
  \choice The frictional force on the crate is zero newtons.
\end{choices}
\vfill{}

\question If the forces that act on a cannonball and the recoiling cannon from which it is fired are equal in magnitude, why do the cannonball and cannon have very different accelerations?
\begin{choices}
  \choice The cannon pushes on the cannon ball with a much greater force than the cannon ball pushes on the cannon.
  \choice Remember $F=m\,a$, and note that the cannon has a much greater mass than the cannon ball, so the cannon accelerates less for the same force.
  \choice The force from the cannon ball on the cannon produces little acceleration because it lasts for such a brief time.
  \choice The cannon weighs so much that friction with the ground alone prevents it from accelerating.
\end{choices}
\vfill{}

\question A pair of physics lab carts, one half the mass of the other, fly apart when a compressed spring that joins them is released. The spring exerts the greater force on the
\begin{choices}
  \choice lighter car.
  \choice same on each.
  \choice heavier car.
\end{choices}
\vfill{}

\newpage{}

\question It is just as difficult to accelerate a car on a level surface on the Moon as it is here on Earth because
\begin{choices}
  \choice the weight of the car is independent of gravity.
  \choice the mass of the car is independent of gravity.
  \choice both of these
  \choice neither of these
\end{choices}
\vfill{}

\question Which has a greater momentum: a heavy truck at rest or a moving skateboard?
\begin{choices}
  \choice The heavy truck at rest.
  \choice The moving skateboard.
  \choice Both have zero momentum.
  \choice Both have the same momentum.
\end{choices}
\vfill{}

\question
What is the momentum of an 8.4-kg baby crawling at 2.2 m/s? (Include the appropriate units in your answer.)

\vfill{}

\question When the speed of a puppy is doubled, how much more kinetic energy does it have?
\begin{choices}
  \choice It has the same kinetic energy.
  \choice It has twice as much.
  \choice It has four times as much.
  \choice It has half as much.
\end{choices}
\vfill{}

\question
How much work is done when you push a crate horizontally with 150 N across a 8.0-m factory floor? (Include the appropriate unit in your answer.)

\vfill{}

\newpage{}

\question How much energy is given to each coulomb of charge that flows through a 1.5-V battery?
\begin{choices}
  \choice 1.5 joules
  \choice 0.66 watts
  \choice 0.66 joules
  \choice 1.5 watts
\end{choices}
\vfill{}

\question In a circuit of two lamps in parallel, where there is a voltage of 120 V across one lamp, what is the voltage across the other lamp?
\begin{choices}
  \choice 120 volts
  \choice 60 volts
  \choice -120 volts
  \choice 90 volts
\end{choices}
\vfill{}

\question When the filament breaks in one lamp in a series circuit, other lamps in the circuit normally
\begin{choices}
  \choice go out
  \choice continue glowing, but dimmer
  \choice continue glowing as brightly
  \choice absorb energy from the damaged lamp
\end{choices}
\vfill{}

\question Current in a conductor can be decreased by
\begin{choices}
  \choice increasing its resistance.
  \choice decreasing the voltage across it.
  \choice both of these
  \choice neither of these
\end{choices}
\vfill{}

\question Just as in hydraulic circuits there is water pressure, in electric circuits there is
\begin{choices}
  \choice current.
  \choice resistance.
  \choice voltage.
\end{choices}
\vfill{}

\newpage{}

\question What is a transistor composed of, and what are some of its functions?
\begin{choices}
  \choice A transistor is made of layers of conductors and insulators sandwiched together. It is a capacitor and stores charge.
  \choice A transistor is made of thin layers of semiconducting materials sandwiched together. It can be a switch, an oscillator, or an amplifier.
  \choice A transistor is made of layers of conductors and insulators sandwiched together. It can be a switch, an oscillator, or an amplifier.
  \choice A transistor is made of thin layers of semiconducting materials sandwiched together. It is a source of energy.
\end{choices}
\vfill{}

\question What produces a magnetic field?
\begin{choices}
  \choice Electric dipoles
  \choice Electric charges in motion
  \choice Voltage
  \choice Magnetic monopoles
\end{choices}
\vfill{}

\question Why does a transformer require ac?
\begin{choices}
  \choice Transformers are designed to use ac because that is the common power source for houses in the U.S.
  \choice A voltage is induced only when the magnetic field is changing.
  \choice No dc current can flow through the primary coil of wire.
  \choice Flux capacitors require ac to power DeLorean-based time machines.
\end{choices}
\vfill{}

\question How are frequency and period related to each other?
\begin{choices}
  \choice The period is the square of the frequency.
  \choice Frequency and period are directly proportional.
  \choice Frequency is the square of the period.
  \choice Frequency and period are reciprocals.
\end{choices}
\vfill{}

\question Why do the same notes plucked on a banjo and on a guitar have distinctly different sounds?
\begin{choices}
  \choice The banjo produces longitudinal sound waves, whereas the guitar produces transverse sound waves.
  \choice The strings have different lengths on a banjo and guitar.
  \choice The hole in a guitar makes it a wind instrument, while the skin surface of the banjo makes it a percussion instrument.
  \choice The banjo and guitar have different sound spectra when playing the same note, so they have a different sound, too.
\end{choices}
\vfill{}

\newpage{}

\question Why do most alpha particles fired through a piece of gold foil emerge almost undeflected?
\begin{choices}
  \choice The alpha particles behave like waves and pass right around the gold atoms.
  \choice The gold atoms behave like waves, so the alpha particles pass right through.
  \choice The massive alpha particles blast through the majority of the space in the gold that is occupied by low mass electrons.
  \choice Alpha particles are small and dense, whereas the gold atoms are bloated and uniformly low density, so the alpha particles blast right through.
\end{choices}
\vfill{}

\question What is a quantum of light called?
\begin{choices}
  \choice A photon
  \choice An electron
  \choice A proton
  \choice A magnetron
\end{choices}
\vfill{}

\question What classical idea about space and time did Einstein reject?
\begin{choices}
  \choice Einstein rejected the idea that space and time are coupled.
  \choice Einstein rejected the idea that the speed of light in a vacuum is the same in all reference frames.
  \choice Einstein rejected the idea that space and time are independent.
  \choice Einstein rejected the idea that time is influenced by gravity.
\end{choices}
\vfill{}

\question Time is required for light to travel along a path from one point to another. If this path is seen to be longer because of motion, what happens to the time it takes for light to travel this longer path?
\begin{choices}
  \choice The time is always the same, regardless of path length.
  \choice It takes a shorter time.
  \choice It takes longer.
  \choice The time is zero.
\end{choices}
\vfill{}

\question Do the relativity equations for time, length, and momentum hold true for everyday speeds? Explain.
\begin{choices}
  \choice Relativity equations do not apply. They only apply at speeds near the speed of light.
  \choice They hold true but the differences they predict are hard to measure.
  \choice Relativity equations do apply, but they are so hard to use that we use the simpler Newtonian equations even though they are measurably wrong.
  \choice Relativity equations do not apply at everyday speeds. The equations are different from Newton's equations at all speeds.
\end{choices}
\vfill{}

\newpage{}

\question Which scientist invented the chemical battery?
\begin{choices}
  \choice Faraday
  \choice Henry
  \choice Maxwell
  \choice Volta
\end{choices}
\vfill{}

\question In what year did Rutherford discover the atomic nucleus?
\begin{choices}
  \choice 1899
  \choice 1905
  \choice 1911
  \choice 1926
\end{choices}
\vfill{}

\question In what year did Faraday discover induction?
\begin{choices}
  \choice 1800
  \choice 1831
  \choice 1865
  \choice 1885
\end{choices}
\vfill{}

\question In what year did Kepler publish his discoveries about planetary motion?
\begin{choices}
  \choice 1543
  \choice 1609
  \choice 1687
  \choice 1800
\end{choices}
\vfill{}

\question List the discoveries in chronological order:
\begin{choices}
  \choice Einstein discovers relativity, Galileo discovers inertia, Maxwell unifies electromagnetism. 
  \choice Einstein discovers relativity, Maxwell unifies electromagnetism, Galileo discovers inertia. 
  \choice Galileo discovers inertia, Einstein discovers relativity, Maxwell unifies electromagnetism. 
  \choice Galileo discovers inertia, Maxwell unifies electromagnetism, Einstein discovers relativity.
  \choice Maxwell unifies electromagnetism, Einstein discovers relativity, Galileo discovers inertia. 
  \choice Maxwell unifies electromagnetism, Galileo discovers inertia, Einstein discovers relativity.
\end{choices}

\end{questions}
\end{document}
